


\documentclass[journal,12pt,onecolumn,draftclsnofoot]{IEEEtran}


\usepackage{amsthm}
\usepackage{amsmath}
\usepackage{amssymb}
\usepackage{booktabs}
\usepackage{color}
\usepackage{epsfig}
\usepackage{subfigure}
\usepackage{graphicx}
%\usepackage{caption}
\usepackage{colortbl}

\renewcommand{\arraystretch}{1.3} 
\newcommand{\diag}{\mathop{\mathrm{diag}}}
\newcommand{\tabincell}[2]{\begin{tabular}{@{}#1@{}}#2\end{tabular}}  

\newtheorem{assumption}{\textbf{Assumption}}
\newtheorem{definition}{\textbf{Definition}}
\newtheorem{lemma}{\textbf{Lemma}}
\newtheorem{theorem}{\textbf{Theorem}}
\newtheorem{remark}{\textbf{Remark}}
\newtheorem{proposition}{\textbf{Proposition}}
\newtheorem{corollary}{\textbf{Corollary}}

\def\ba{\begin{array}}
	\def\ea{\end{array}}
\newcommand{\beq}{\begin{equation}}
\newcommand{\eeq}{\end{equation}}
\newcommand{\bq}{\begin{eqnarray}}
\newcommand{\eq}{\end{eqnarray}}
\newcommand{\bqn}{\begin{eqnarray*}}
	\newcommand{\eqn}{\end{eqnarray*}}
\newcommand{\bee}{\begin{enumerate}}
	\newcommand{\eee}{\end{enumerate}}
\newcommand{\bi}{\begin{itemize}}
	\newcommand{\ei}{\end{itemize}}
%\newcommand{\qed}{\hfill{$\blacksquare$}}
\newcommand{\ii}{\textbf{i}}

\usepackage{comment}
%\newboolean{showcomments}
%\setboolean{showcomments}{true}
\newcommand{\slow}[1]{\ifthenelse{\boolean{showcomments}}
	{ \textcolor{red}{(SL:  #1)}}{}}
\newcommand{\you}[1]{\ifthenelse{\boolean{showcomments}}
	{ \textcolor{green}{(PCY:  #1)}}{}}
\newcommand{\john}[1]{\ifthenelse{\boolean{showcomments}}
	{ \textcolor{blue}{(jp:  #1)}}{}}


% correct bad hyphenation here
\hyphenation{op-tical net-works semi-conduc-tor}


\begin{document}

%\title{Question}


%\author{Pengcheng You \\  Nov 14 2017% <-this % stops a space
%\thanks{M. Shell was with the Department
%of Electrical and Computer Engineering, Georgia Institute of Technology, Atlanta,
%GA, 30332 USA e-mail: (see http://www.michaelshell.org/contact.html).}% <-this % stops a space
%\thanks{J. Doe and J. Doe are with Anonymous University.}% <-this % stops a space
%\thanks{Manuscript received April 19, 2005; revised August 26, 2015.}
%}


% make the title area
%\maketitle


%\begin{abstract}
%
%\end{abstract}

%
%\begin{IEEEkeywords}
%
%\end{IEEEkeywords}


\IEEEpeerreviewmaketitle
For each bus $j$, the nonlinear dynamical model is as follows \cite{wang2017distributed,stegink2017unifying,stegink2016stabilization}:
%\begin{subequations}
\begin{small}
\begin{eqnarray*}
\dot \theta_j & = & \omega_j   \\
M_j \dot \omega_j & = & r_j+p_j-d_j - D_j \omega_j -\sum_{k:j\rightarrow k}  {\color{red}   E'_{qj} E'_{qk}    }   B_{jk} \sin (\theta_j-\theta_k) +\sum_{i:i\rightarrow j}   {\color{red}   E'_{qi} E'_{qj}       }    B_{ij} \sin (\theta_i-\theta_j)   \\
0 & = & r_j-d_j - D_j \omega_j -\sum_{k:j\rightarrow k}  {\color{red}   E'_{qj} E'_{qk}    }   B_{jk} \sin (\theta_j-\theta_k) +\sum_{i:i\rightarrow j}   {\color{red}   E'_{qi} E'_{qj}       }    B_{ij} \sin (\theta_i-\theta_j)   \\
0 & = & r_j- D_j \omega_j -\sum_{k:j\rightarrow k}  {\color{red}   E'_{qj} E'_{qk}    }   B_{jk} \sin (\theta_j-\theta_k) +\sum_{i:i\rightarrow j}   {\color{red}   E'_{qi} E'_{qj}       }    B_{ij} \sin (\theta_i-\theta_j)   \\
T'_{dj}    \dot E'_{qj} & = &  {\color{red}  E_{fj}   }  -  \left[1-({\color{red} x_{dj}}- {\color{red}x'_{dj}}) {\color{red} B_{jj}  } \right] E'_{qj}  +  (x_{dj}-x'_{dj})    \sum_{k\in \mathcal{N}(j)}  E'_{qk} B_{jk} \cos (\theta_j-\theta_k)   
\end{eqnarray*}	\end{small}%
%\end{subequations}%
where $E'_{qj}$ is the $q$-axis transient internal voltage, $T'_{dj}$ is the $d$-axis transient time constant, $E_{fj}$ is the constant excitation voltage set to $1$, $x_{dj}$ and $x'_{dj}$ are the $d$-axis synchronous and transient reactances, respectively, and $\mathcal{N}(j)$ denotes the set of neighbor buses that are connected to bus $j$.  











\bibliographystyle{IEEEtran}  
\bibliography{bib}  


\end{document}


